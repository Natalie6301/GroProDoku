\chapter{Benutzeranleitung}\label{ch:benutzeranleitung}


\section{Vorbereiten des Systems}\label{sec:vorbereiten-des-systems}

\subsection{Systemvoraussetzungen}\label{subsec:systemvoraussetzungen}
Um das Programm verwenden zu können wird Windows benötigt, bei der Entwicklung wurde mit Windows 10 gearbeitet, daher ist das auch die Empfehlung für die Nutzung.
Zusätzlich wird eine Java Runtime Environment benötigt.

\subsection{Installation}\label{subsec:installation}
Das Projekt ist in einem Zip-komprimierten Ordner.
Um die Software nutzen zu können, muss diese entpackt werden.


\section{Programmaufruf}\label{sec:programmaufruf}
Die ausführbare Datei heißt GroProSim.jar und kann über Command-Line genutzt werden.
Hierfür wird "java -jar GroProSim.jar <Dateiname>.txt" ausgeführt.
Der Dateiname ist der Name der Datei, in dem die Eingabedaten im richtigen Format geschrieben sind.
Die Ausgabe wird auf der Konsole erscheinen und die erstellten Dateien sind im Ordnerverzeichnis, von der aufgerufen wurde, hinterlegt.


\section{Testen der Beispiele}\label{sec:testen-der-beispiele}
Die Testdaten aus der Klausur sind als Textdateien im Resource-Ordner enthalten.