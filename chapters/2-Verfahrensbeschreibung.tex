\chapter{Verfahrensbeschreibung}\label{ch:verfahrensbeschreibung}

\section{Gesamtsystem}\label{sec:gesamtsystem}
Die entwickelte Software liest die Daten ein und führt die Prüfungen nach semantischen und syntaktischen Fehlern aus.
Danach ermittelt der Algorithmus die minimale Anzahl an Antennen, die benötigt wird, um alle Eckpunkte jeder Teilfläche zu erreichen.
Im Anschluss werden die Daten in der Konsole wiedergegeben und in eine Datei geschrieben, sowie die Anweisungen für GnuPlot geschrieben.

\subsection{Eingabe}\label{subsec:eingabe}
Beim Einlesen wird der Pfad, der eingegeben wird, genutzt und die Datei gelesen.
Nach allen syntaktischen und semantischen Überprüfungen werden die Variablen mit den Daten gefüllt.
Hierbei wird für jeden Höhenwert ein Punkt erstellt, der die x-, y- und z-Koordinate füllt.

\subsection{Verarbeitung}\label{subsec:verarbeitung}
Die Verarbeitung enthält den Algorithmus, der in mehrere Schritte aufgeteilt ist und dementsprechend unterschiedliche Methoden nutzt.
Der Algorithmus startet mit einer Antenne un prüft, ob es einen Punkt gibt, der alle anderen Punkte erreicht.
Danach werden alle Kombinationen errechnet, die zwei Antennen enthalten.
Diese Möglichkeiten werden iteriert und geschaut, ob mit dieser Kombination alle Eckpunkte erreichbar sind.
Wenn immer noch nicht alle Punkte erreicht wurden, werden drei Antennen gesetzt.
Dieses Verfahren nimmt nach jeder Iteration eine weitere Antenne dazu.
Sobald eine Möglichkeit gefunden wurde, alle Punkte zu erreichen, ist das die minimale Anzahl an Antennen.

\subsection{Ausgabe}\label{subsec:ausgabe}
Die Ausgabe erfolgt, sobald die Anzahl an Antennen gefunden wurde.
Zunächst werden die Eingabedaten wiederholt und die Antennenanzahl mit ihren Koordinaten in die Konsole ausgegeben und in eine Datei geschrieben.
Danach werden die Anweisungen für GnuPlot zusammengesetzt.
Für GnuPlot werden weitere Dateien benötigt, die die Koordinaten der Eckpunkte der Teilflächen enthält und jeweils eine Datei mit Koordinaten für jede Antennen.

\subsection{Datenstrukturen}\label{subsec:datenstrukt}
Innerhalb des Programms sind mehrere Modelle erstellt, um die Daten zu speichern und die Berechnung zu ermöglichen.
Das erste Modell Punkt hält die Koordinaten der einzelnen Höhen mit x- und y-Koordinate.
Außerdem enthält dieses Modell Methoden, um die Erreichbarkeit zwischen zwei Punkten zu prüfen.
Des Weiteren wird diese Klasse genutzt, um das nächste Modell aufzustellen.
Das Modell Gerade enthält zwei Vektoren, die als Punkt gespeichert werden.
Die beiden Punkte entsprechen dem Stütz- und Richtungsvektor, die eine Gerade im dreidimensionalen System aufspannen.
Das dritte genutzte Modell ist die Klasse Ebene, diese wird aus drei Vektoren erzeugt.
Hierbei ist der erste Punkt der Aufpunkt, bei dem die Ebene ansetzt.
Der zweite und dritte Punkt wird genutzt, um die Richtungsvektoren zu berechnen.
Die Logik des Algorithmus liegt im Controller Antennenminimum.
Die dritte Schicht des MVC-Systems ist View, die vom Nutzer den Pfad übergeben bekommt.