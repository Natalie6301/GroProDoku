\chapter{Aufgabenanalyse}\label{ch:aufgabenanalyse}


\section{Interpretation der Aufgabe}\label{sec:interpretation-der-aufgabe}
Im Flächengebiet der Matsedanier sollen Antennen aufgestellt werden, sodass jeder Eckpunkt der Teilflächen erreicht wird, da davon ausgegangen wird, wenn die Eckpunkte erreicht werden, wird die komplette Teilfläche erreicht.
Um so wenig wie mögliche Antennen aufzustellen, soll ein Programm entwickelt werden, welches überprüft, wie viele minimal aufgestellt werden müssen.
Hierfür wird überprüft, dass eine Antenne möglichst viele Punkte erreicht und dabei keine andere Fläche durchdringt.
\\
Das Programm arbeitet hierfür mit dem Eingabe-Verarbeitung-Ausgabe-Prinzip.
Zunächst werden die Daten für das Flächengebiet eingelesen.
Die Datei enthält den Namen des Beispiels, die Größen des zu berechnenden Bereiches und die Höhenangaben an den Eckpunkten.
Es muss mindestens eine Angabe für die Höhe in der Zeile sein, da dann von ausgegangen wird, dass die nachfolgenden Höhen der vorangestellten entsprechen.
Nachdem das Programm die Mindestanzahl an Antennen berechnet hat, werden mindestens vier Ausgabedateien erzeugt.
Die Erste enthält die Eingabewerte, die Antennenanzahl und die Koordinaten der einzelnen Antennen.
Die Zweite enthält die Anweisungen für GnuPlot, die Dritte enthält die Eckpunktkoordinaten der Teilflächen und die Vierte und Folgenden enthalten die Koordinaten der Antennen.
\\
Im Programm werden die eingelesenen Daten in einer Klasse Punkt gespeichert, diese enthält die drei Koordinaten des Eckpunktes.
Im Vergleich zur Überlegung an Tag 1 ergibt diese Art des Speicherns der Daten den Vorteil, die Daten gekapselt zu nutzen und sortiert zu speichern.
Hierbei sind die Koordinaten als Double gespeichert, obwohl die Koordinaten keine Nachkommastellen besitzen, da die Schnittpunkte bei der Berechnung der Erreichbarkeit Nachkommastellen besitzen können.
\\
Beim Einlesen werden die Daten überprüft, unter anderem wird die Anzahl der Daten überprüft, ob diese der Größe des Feldes entsprechen.
Wenn die x-Koordinaten-Anzahl nicht mit der angegebenen Größe übereinstimmt, werden diese mit dem letzten Wert aufgefüllt.
Falls die y-Koordinaten-Anzahl nicht übereinstimmt, wird abgebrochen und eine Exception geworfen.
\\
Der Algorithmus, der am ersten Tag beschrieben wurde, war schwer als Programm umzusetzen, sodass ein neuer Algorithmus entwickelt wurde.
Der neue Algorithmus überprüft jede Möglichkeit eine einzelne Antenne aufzustellen, sodass jeder andere Punkt erreicht werden kann.
Hierbei startet die Überprüfung der möglichen Positionen bei dem höchstmöglichen Punkt und geht die Liste der Datenpunkte absteigend der Höhen entlang.
Falls nicht alle Punkte erreicht werden können, werden alle Möglichkeiten zwei Antennen aufzustellen überprüft.
Hierbei werden die Datenpunkte wieder absteigend der Höhen kombiniert.
Der Algorithmus überprüft danach jede Dreier-Kombinationen, Vierer usw.
Hierbei auch wieder der Höhe absteigend nach sortiert.
Der Algorithmus bricht ab, sobald alle Datenpunkte durch eine Antenne abgedeckt sind.
\\
Die Ausgabe passiert, nachdem die kleinste Anzahl an Antennen gefunden wurde.
Zunächst werden in der Konsole die Eingabedaten wiederholt und das Ergebnis der Antennenanzahl mit den Koordinaten aufgelistet.
Die erste Datei, die erzeugt wird, enthält ebenfalls diese Daten, die ausgegeben wurden.
Danach werden die Dateien für GnuPlot erzeugt.
Die erste Datei enthält die Anweisungen, um das Koordinatenfeld, die Felderflächen und Antennen zu setzen.
Die nächste Datei enthält die Koordinaten, die die Eckpunkte beschreiben.
Nachfolgend werden die Dateien für die Koordinaten der Antennen erzeugt.
Pro Datei werden die Koordinaten angegeben, die die Antennen aufspannen, da diese 10 Meter hoch ist.



\section{Fehlerarten}\label{sec:fehlerarten}

\subsection{Technische Fehler}\label{subsec:technische-fehler}
Ein technischer Fehler in dieser Software tritt auf, falls die einzulesende Datei nicht gefunden werden kann.
Das tritt auf, falls die Datei im falschen Ordner liegt und/oder nicht der gesamte Pfad zu dieser Datei eingegeben wird.

\subsection{Syntaktische Fehler}\label{subsec:syntaktische-fehler}
Beim Einlesen der Datei können viele Fehler auftreten.
Die Datei muss mindestens sieben Zeilen enthalten.
Hierbei werden die Kommentarzeilen mit gezählt und geprüft.
Die erste Zeile der Datei muss eine Kommentarzeile sein, also mit einem Semikolon starten.
Des Weiteren muss die erste Zeile "*" beinhalten.
Die zweite Zeile ist eine Kommentarzeile, die den Titel des Beispiels enthält.
Die dritte Zeile wird wie die erste überprüft.
Nachfolgend wird die nächste Zeile geprüft, dass diese wieder eine Kommentarzeile ist.
Die sechste Zeile muss auch eine Kommentarzeile sein.
Als Nächstes wird die Dimension in y-Richtung überprüft, ob diese der angegebenen Dimension entspricht.
Hierbei werden die Zeilen nach der letzten Kommentarzeile gezählt und mit der angegebenen Größe des Feldes verglichen.
Die Höhen und Dimensionen, die angegeben werden, dürfen keine Nachkommastellen haben.
Beim Schreiben der Ausgabe, kann es passieren, dass die Daten nicht in die Dateien geschrieben werden können.

\subsection{Semantische Fehler}\label{subsec:semantische-fehler}
In der fünften Zeile sollen die Dimensionen des gesamten Feldes angegeben werden, falls diese negativ oder null sind, tritt ein Fehler auf.
Des Weiteren dürfen angegebenen Höhen nur zwischen null und sechs gesetzt sein.

\section{Fehlerbehandlung}\label{sec:fehlerbehandlung}

\subsection{Technische Fehler}\label{subsec:technische-fehler-behandlung}
Das Programm bricht ab und kann mit dem richtigen Dateipfad/-namen neu gestartet werden.

\subsection{Syntaktische Fehler}\label{subsec:syntaktische-fehler-behandlung}
Jeder syntaktische Fehler hat seine eigene Fehlernachricht, die an den Nutzer weitergegeben wird.
Das Programm bricht ab.

\subsection{Semantische Fehler}\label{subsec:semantische-fehler-behandlung}
Es werden auch bei semantischen Fehlern Nachrichten an den Nutzer zurückgegeben und das Programm beendet sich.

\subsection{Sonderfälle}\label{subsec:sonderfaelle}
Wenn die Anzahl pro Zeile nicht mit der angegebenen x-Dimension des Feldes überein stimmt, werden die Werte mt dem letzten Wert aufgefüllt.
